Consider the following function
$$
u(x) = \frac{1}{|x|^\gamma}
$$
in $\domain = B_1(0)$.

Show that if $\gamma + 1 < n$, the weak derivatives are given by
$$
\partial_j u = -\gamma \frac{x_j}{|x|^{\gamma + 2}}
$$
i.e., you need to show rigorously that
$$
\int u \partial_j \phi = \int \phi \gamma \frac{x_j}{|x|^{\gamma + 2}}
$$

Find the condition on $\gamma$ s.t. $u \in W^{1, p}$ and $u \in W^{2, p}$

For $u \in W^{1, p}$, we just follow Evans' example 3.
This comes out to $\gamma < \frac{n - p}{p}$.

First we note that $u$ is smooth away from $0$, with
$u_{x_i}(x) = \frac{-\gamma x_i}{|x|^{\gamma + 2}}$, so
$|D u(x)| = \frac{\gamma}{|x|^{\gamma + 1}}$
for any distance at least $\epsilon > 0$ from the origin.

Then, for any $\phi \in C^\infty_C(\domain)$, the weak derivative $v$ satisfies
\begin{align*}
  \int_{\domain} u \phi_{x_i} \dx = &-\int_{\domain} v \phi \dx\\
  \int_{\epsilon < |x| < 1} u \phi_{x_i} \dx = &-\int_{\epsilon < |x| < 1} v \phi \dx
\end{align*}
Integrating by parts gives
\begin{align*}
  \int_{\epsilon < |x| < 1} u \phi_{x_i} \dx = &\int_{\bndry{B_\epsilon(0)}} u \phi \nu^i \dd{S}
                                        - \int_{\epsilon < |x| < 1} u_{x_i} \phi \dx
\end{align*}
where $\nu$ is the inward normal of $\bndry{B_\epsilon(0)}$.

Since $u$ and $\phi$ are bounded on the boundary, we have
\begin{align*}
\left| \int_{\bndry{B_\epsilon(0)}} u \phi \nu^i \dd{S} \right|
  \leq &\int_{\bndry{B_\epsilon(0)}} \epsilon^{-\gamma} \dd{S} \leq C \epsilon^{n - 1 - \alpha}\\
  \leq &||\phi||_{L^\infty} \int_{\bndry{B_\epsilon}} \epsilon^{-\gamma} \dd{S} \leq C \epsilon^{n - 1 - \alpha}
\end{align*}
This goes to $0$ as $\epsilon \rightarrow 0$ if $n - 1 - \alpha > 0$.

Then
\begin{align*}
  \limitto{\epsilon}{0} \int_{\epsilon < |x| < 1} u \phi_{x_i} \dx
    = &\limitto{\epsilon}{0} -\int_{\epsilon < |x| < 1} u_{x_i} \phi \dx\\
  \int_{B_1(0)} u \phi_{x_i} \dx = &\int_{|x| < \epsilon} u \phi_{x_i} \dx + \int_{\epsilon < |x| < 1} u \phi_{x_i} \dx\\
    = &\limitto{\epsilon}{0} \int_{|x| < \epsilon} u \phi_{x_i} \dx - \int_{\epsilon < |x| < 1} u_{x_i} \phi \dx\\
    = &\limitto{\epsilon}{0} -\int_{\epsilon < |x| < 1} u_{x_i} \phi \dx
\end{align*}
proving that this weak derivative works over the entire domain excluding $0$ when $\gamma + 1 < n$.

Next, we note that the weak derivative is the same form as the original; just multiplied by a constant.
Then applying the same process above, except starting with $\gamma + 1$, for the second weak derivative
we end up with $\gamma + 2 < n$, with
\begin{align*}
|D^2 u(x)| = \frac{|\gamma (\gamma + 1)|}{|x|^{\gamma + 2}}
\end{align*}

Finally, we need to determine when the $L^p$ norms exist for all of these.
Again, for any $0 < \epsilon < 1$, with $n - p (\gamma + 2) > 0$, we have
\begin{align*}
||u||_{L^p} = &\int_{B_1(0)} |x|^{-p \gamma} \dx
  = \int_{r < \epsilon} C r^{n - 1 - p \gamma} \dr + \int_{\epsilon < r < 1} C r^{n - 1 - p \gamma} \dr\\
  = &\limitto{\epsilon}{0} \int_{\epsilon < r < 1} C r^{n - 1 - p \gamma} \dr
  = \limitto{\epsilon}{0} C r^{n - p \gamma} \big|_{r = \epsilon}^{1}
  = C
\end{align*}

\begin{align*}
||D u||_{L^p} = &\int_{B_1(0)} C |x|^{-p (\gamma + 1)} \dx
  = \int_{r < \epsilon} C r^{n - 1 - p (\gamma + 1)} \dr + \int_{\epsilon < r < 1} C r^{n - 1 - p (\gamma + 1)} \dr\\
  = &\limitto{\epsilon}{0} \int_{\epsilon < r < 1} C r^{n - 1 - p (\gamma + 1)} \dr
  = \limitto{\epsilon}{0} C r^{n - p (\gamma + 1)} \big|_{r = \epsilon}^{1}
  = C
\end{align*}

\begin{align*}
||D^2 u||_{L^p} = &\int_{B_1(0)} C |x|^{-p (\gamma + 2)} \dx
  = \int_{r < \epsilon} C r^{n - 1 - p (\gamma + 2)} \dr + \int_{\epsilon < r < 1} C r^{n - 1 - p (\gamma + 2)} \dr\\
  = &\limitto{\epsilon}{0} \int_{\epsilon < r < 1} C r^{n - 1 - p (\gamma + 2)} \dr
  = \limitto{\epsilon}{0} C r^{n - p (\gamma + 2)} \big|_{r = \epsilon}^{1}
  = C
\end{align*}
Thus, this function is in $W^{2, p}(\domain)$ whenever $n - p (\gamma + 2) > 0$,
since that is the most restrictive bound on it (as $p > 0$).
