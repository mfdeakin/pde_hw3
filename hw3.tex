
\documentclass[letterpaper,11pt]{article}

\usepackage{latexsym}
\usepackage{amsmath}
\usepackage{amssymb}
\usepackage{fancyhdr}
\usepackage[margin=1.0in, left=0.5in, right=0.5in, top=1.25in, headsep=10mm, headheight=15mm]{geometry}
\usepackage{graphicx}

\pagestyle{fancy}
\rhead{Michael Deakin\\Math 516\\Homework 3}

\newcommand*\limitset[1]{{#1}^\prime}
\newcommand*\closure[1]{\overline{#1}}
\newcommand*\closureunion[1]{{#1}\cup \limitset{#1}}
\newcommand*\interior[1]{{#1}^\circ}
% The set of points within some distance #1 from #2
\newcommand*\neighbor[2]{N_{#1}({#2})}
% The neighborhood without #2
\newcommand*\delneighbor[2]{N_{#1}^*({#2})}
\newcommand*\set[1]{\{ #1 \} }
\newcommand*\conjugate[1]{\overline{#1}}
\newcommand*\sequence[2]{\set{#1}_{#2=1}^\infty}
\newcommand*\series[2]{\sum_{#2=1}^\infty #1_{#2}}
\newcommand*\compose[2]{#1 \circ #2}
\newcommand*\udisk[0]{\mathbb{D}}
\newcommand*\disk[2]{D_{#1}(#2)}
\newcommand*\punctdisk[2]{\disk_{ #1 } - \set{#2}}
\newcommand*\complex[0]{\mathbb{C}}
\newcommand*\naturals[0]{\mathbb{N}}
\newcommand*\rationals[0]{\mathbb{Q}}
\newcommand*\reals[0]{\mathbb{R}}

\newcommand*\domain[0]{\Omega}
\newcommand*\bndry[1]{\partial #1}
\newcommand*\bndrydom[0]{\partial \domain}
\newcommand*\compactcont[0]{\subset \subset} % U \compactcont V \rightarrow U \subset \closure{U} \subset V, where U, V are (open) domains

\newcommand*\ball[2]{B_{#2}(#1)}

\newcommand*\limitto[2]{\lim \limits_{#1 \rightarrow #2}}

\newcommand{\dd}[1]{\;\mathrm{d}#1}
\newcommand{\dx}{\dd{x}}
\newcommand{\dy}{\dd{y}}
\newcommand{\dz}{\dd{z}}
\newcommand{\dr}{\dd{r}}
\newcommand{\ds}{\dd{s}}
\newcommand{\dt}{\dd{t}}
\newcommand*\pderiv[2]{\frac{\partial #1}{\partial #2}}
\newcommand*\nthpderiv[3]{\frac{\partial^{#3} #1}{\partial #2^{#3}}}
\newcommand*\deriv[2]{\frac{\dd{#1}}{\dd{#2}}}
\newcommand*\nthderiv[3]{\frac{\dd{^{#3} #1}}{\dd{#2^{#3}}}}

\DeclareMathOperator{\res}{res}
\DeclareMathOperator{\sign}{sign}
\DeclareMathOperator{\diam}{diam}
\DeclareMathOperator{\partition}{Partition}

% Average integral from https://tex.stackexchange.com/questions/759/average-integral-symbol
\def\Xint#1{\mathchoice
{\XXint\displaystyle\textstyle{#1}}%
{\XXint\textstyle\scriptstyle{#1}}%
{\XXint\scriptstyle\scriptscriptstyle{#1}}%
{\XXint\scriptscriptstyle\scriptscriptstyle{#1}}%
\!\int}
\def\XXint#1#2#3{{\setbox0=\hbox{$#1{#2#3}{\int}$ }
\vcenter{\hbox{$#2#3$ }}\kern-.6\wd0}}
\def\ddashint{\Xint=}
\def\dashint{\Xint-}
\def\avgint{\dashint}

\begin{document}

I collaborated with Damien Huet on this assignment

\begin{enumerate}
\item Derive a solution formula for the two-dimensional wave equation with source
$$
\begin{cases}
  u_{tt} = \Delta u + f(x, t) & x \in R^2\\
  u(x, 0) = 0, u_t(x, 0) = 0
\end{cases}
$$

We apply Duhamel's principle.
Define $v(x, t; s)$ s.t. it satisfies the following
$$
\begin{cases}
  v_{tt}(x, t; s) - \Delta v(x, t; s) = 0\\
  v(x, s; s) = 0\\
  v_t(x, s; s) = f(x, s)
\end{cases}
$$
Applying Poisson's formula to compute a solution gives
$$
v(x, t; s) =\frac{1}{2} \avgint_{B_t(x)} \frac{t^2 f(y; s)}{\sqrt{t^2 - |y - x|^2}} \dy
$$

Then by Duhamel's principle
$$
u(x, t) = \int_0^t v(x, t; s) \ds = \frac{1}{2} \int_0^t \avgint_{B_t(x)} \frac{t^2 f(y; s)}{\sqrt{t^2 - |y - x|^2}} \dy \ds
$$

Simplifying beyond this doesn't seem possible...

\item Derive a solution formula for the three-dimensional wave equation with radial source
$$
\begin{cases}
  u_{tt} = \Delta u + f(r, t) & t > 0, r > 0\\
  u(r, 0) = 0, u_t(r, 0) = 0
\end{cases}
$$

Again we start with Duhamel's Principle, and define $v$ s.t. it satisfies the following system:
$$
\begin{cases}
  v_{tt}(x, t; s) = \Delta v(x, t; s) & t > 0\\
  v(x, 0; s) = 0, v_t(x, 0; s) = f(|x|, s)
\end{cases}
$$

Then since $v$ is three-dimensional, it is solved with Kirchoff's Formula:
$$
v(x, t; s) = \avgint_{\partial B_t(x)} t f(|y|; s) \dd{S(y)}
$$

Next, we consider the intersection of $\partial B_t(x)$ with $\partial B_r(0)$ for some $r \in [\max(|x| - t, t - |x|), |x| + t]$.
This intersection is a circle.
Since $f$ is radially symmetric, it's constant over this circle.
So we can rewrite the previous integral by computing the derivative of the area of the spherical cap
with respect to the distance $r$ of its boundary circle to the origin.
Recall that the area of the spherical cap is given by $\pi (a^2 + h^2)$,
where $a$ is the (minimum) distance from the $Ox_0$ axis to the boundary circle,
and $h$ is the distance from the plane containing the boundary circle to the axis' intersection with the spherical cap.
On the intersection of these two spheres, we have two cases: $t < |x_0|$ and $t > |x_0|$.
For the former, we compute
\begin{align*}
  t^2 = &(t - h)^2 + a^2 \rightarrow a^2 = t^2 - (t - h)^2\\
  r^2 = &(h + ||x_0| - t|)^2 + a^2 = (h + ||x_0| - t|)^2 + t^2 - (t - h)^2\\
      = &||x_0| - t|^2 + 2 h ||x_0| - t| + 2 t h\\
  h = &\frac{r^2 - ||x_0| - t|^2}{2 ||x_0| - t| + 2 t} = \frac{r^2 - ||x_0| - t|^2}{|x_0|}\\
  A = &\pi (t^2 - (t - h)^2 + h^2) = 2 \pi h t = 2 \pi t \frac{r^2 - ||x_0| - t|^2}{2 |x_0|}
\end{align*}
For the latter, we compute
\begin{align*}
  a^2 = &t^2 - (t - h)^2\\
  r^2 = &(t - |x_0| - h)^2 + a^2 = (t - |x_0| - h)^2 + 2 t h - h^2\\
      = &2 |x_0| h + t^2 + |x_0|^2 - 2 t |x_0|\\
  h = &\frac{r^2 - |x_0|^2 + 2 t |x_0| - t^2}{2 |x_0|}\\
  A = &2 \pi h t = 2 \pi t \frac{r^2 - |x_0|^2 + 2 t |x_0| - t^2}{2 |x_0|}
\end{align*}

Then
$$
dS(y) = \frac{t r}{|x_0|} \dd{\theta} \dr
$$

Since $f$ is constant w.r.t. $\theta$, the surface integral simplifies to
$$
v(x, t; s) = \frac{1}{\alpha_3 t^2} \int \limits_{r = ||x| - t|}^{|x| + t} \frac{\alpha_2 t r}{|x|} f(r; s) \dr
           = \frac{1}{2 t |x|} \int \limits_{r = ||x| - t|}^{|x| + t} r f(r; s) \dr
$$

Then, by Duhmamel's Principle, we have
$$
u(x, t) = \int \limits_{s=0}^t v(x, t; s) \ds
  = \frac{1}{2 t |x|} \int \limits_{s=0}^t \int \limits_{r = ||x| - t|}^{|x| + t} r f(r; s) \dr \ds
$$

\item Find the first order and second order weak derivatives for the following function $u: \reals \rightarrow \reals$, if exists:

\begin{enumerate}
\item
  $$
  u(x) =
  \begin{cases}
    1 - |x| & |x| \leq 1\\
    0 & |x| > 1
  \end{cases}
  $$

  The first order weak derivative is
  $$
  Du(x) =
  \begin{cases}
    1 & -1 < x < 0\\
    -1 & 0 < x < 1\\
    0 & \text{otherwise}
  \end{cases}
  $$
  The second order weak derivative doesn't exist, as it's not continuous (see problem 10a).
  We can prove this by working backwards and using the uniqueness of the weak derivative to prove our claim.
  We need to show that $\int_\reals Du \phi \dx = -\int_\reals u \phi' \dx$ for any $\phi \in C_C^\infty(\reals)$.
  To do so, we simply note that $\deriv{}{x}(1 + x) = 1$ and $\deriv{}{x}(1 - x) = -1$,
  which is nice since we need these to be $0$ at the $-1$ and $1$, respectively, while having the same value at $0$.
  Then integrating by parts results in:
  \begin{align*}
    \int Du(x) \phi(x) = &\int_{-1}^0 \phi(x) \dx - \int_0^1 \phi(x) \dx\\
                       = &\phi(x) (1 + x) \bigg\rvert_{x=-1}^{0} - \int_{-1}^0 (1 + x) \phi'(x)
                          + \phi(x) (1 - x) \bigg\rvert_{x=0}^{1} - \int_0^1 (1 - x) \phi'(x) \dx\\
                       = &\phi(0) - \int_{-1}^0 (1 + x) \phi'(x) - \phi(0) - \int_0^1 (1 - x) \phi'(x) \dx\\
                       = &-\int_{-1}^0 (1 + x) \phi'(x) - \int_0^1 (1 - x) \phi'(x) \dx
                       = -\int_{-1}^0 (1 - |x|) \phi'(x) - \int_0^1 (1 - |x|) \phi'(x) \dx\\
                       = &-\int_{-1}^1 (1 - |x|) \phi'(x)
  \end{align*}

  Thus, $Du$ is the weak derivative of $u$.
  The second order weak derivative does not exist, since $Du(x)$ is discontinuous at $x = 0$ and $x = \pm 1$; see problem 10a.

\item
  $$
  u(x) = |\sin(x)|
  $$

  First we rewrite this in cases:
  $$
  u(x) =
  \begin{cases}
    \sin(x) & 2 k \pi < x < (2 k + 1) \pi\\
    -\sin(x) & (2 k + 1) \pi < x < (2 k + 2) \pi
  \end{cases}
  $$

  Then the first order weak derivative is
  $$
  Du(x) =
  \begin{cases}
    \cos(x) & 2 k \pi < x < (2 k + 1) \pi\\
    -\cos(x) & (2 k + 1) \pi < x < (2 k + 2) \pi
  \end{cases}
  $$

  We check by integrating by parts over all of the subintervals:
  \begin{align*}
    \int_\reals |\sin(x)| \phi'(x) \dx
      = &\sum_{k=-\infty}^\infty \int_{2 k \pi}^{(2 k + 1) \pi} \sin(x) \phi'(x) \dx
         + \int_{(2 k + 1) \pi}^{(2 k + 2) \pi} -\sin(x) \phi'(x) \dx\\
      = &\sum_{k=-\infty}^\infty \sin((2 k + 1) \pi) \phi((2 k + 1) \pi) - \sin(2 k \pi) \phi(2 k \pi)\\
         &- \int_{2 k \pi}^{(2 k + 1) \pi} \cos(x) \phi(x) \dx\\
         &- \sin((2 k + 2) \pi) \phi((2 k + 2) \pi) + \sin((2 k + 1) \pi) \phi((2 k + 1) \pi)\\
         &+ \int_{(2 k + 1) \pi}^{(2 k + 2) \pi} \cos(x) \phi(x) \dx\\
      = &\sum_{k=-\infty}^\infty -\int_{2 k \pi}^{(2 k + 1) \pi} \cos(x) \phi(x) \dx
          + \int_{(2 k + 1) \pi}^{(2 k + 2) \pi} \cos(x) \phi(x) \dx\\
      = &\sum_{k=-\infty}^\infty -\int_{2 k \pi}^{(2 k + 1) \pi} \cos(x) \phi(x) \dx
          - \int_{(2 k + 1) \pi}^{(2 k + 2) \pi} -\cos(x) \phi(x) \dx\\
      = &\int_\reals Du \phi(x) \dx
  \end{align*}
  proving our claim.

  $Du(x)$ is discontinuous at every $x = 2 k \pi$, so the second order weak derivative does not exist.
\end{enumerate}

\item Suppose $u: (a, b) \rightarrow R$ and the weak derivative exists and satisfies
$$
Du = 0, \text{ a.e. in } (a, b)
$$
Prove that $u$ is constant a.e. in $(a, b)$

\item Let $1 < p < \infty$. Show that if $u, v \in W^{1, p}(\Omega)$, then $\max(u, v), \min(u, v) \in W^{1, p}(\Omega)$.
Show that this is not true for $W^{2, p}(\Omega)$.

$u, v \in W^{1, p}(\Omega)$.
First, assuming $D^{x_i} \max(u, v)$ exists, we have $\max(u, v) \in W^{1,p}(\Omega)$, since
\begin{align*}
  \int_\Omega |\max(u, v)|^p + |D^{x_i} \max(u, v)|^p \dx \leq &\int_\Omega |\max(u, v)|^p + \max(|D^{x_i} u|, |D^{x_i} v|)^p \dx\\
  \leq &\int_\Omega |u|^p + |v|^p + |D^{x_i} u|^p + |D^{x_i} v|^p \dx
\end{align*}
is a finite upper bound to these quantities.
By the same argument, we have  $\max(u, v) \in W^{1,p}(\Omega)$.

Next we need to show that these have weak derivatives.
Consider $\int_\Omega \max(u, v) D^{x_i} \phi \dx$ for any $k \leq p$.
Because $u$ and $v$ have weak derivatives up to order $p$,
we know that they are continuous and differentiable almost everywhere, and thus, so is $u - v$.
Then we can partition $\Omega$ into countably many disjoint neighborhoods
$\set{N^u_i}$ in which $u - v > 0$ and $\set{N^v_j}$ in which $u - v < 0$,
with $\closure{\Omega} = \closure{(\cup_i N^u_i \cup_j N^v_j)}$.
Then we can rewrite our integral
\begin{align*}
\int_\Omega \max(u, v) D^{x_k} \phi \dx = &\sum_i \int_{N^u_i} \max(u, v) D^{x_k} \phi \dx
                                         + \sum_j \int_{N^v_i} \max(u, v) D^{x_k} \phi \dx\\
                                      = &\sum_i \int_{N^u_i} u D^{x_k} \phi \dx
                                         + \sum_j \int_{N^v_i} v D^{x_k} \phi \dx
\end{align*}
and noting that since $u$ and $v$ have weak derivatives in $\Omega$,
and therefore in any neighborhood contained in $\Omega$, we have
\begin{align*}
\int_\Omega \max(u, v) D^{x_k} \phi \dx
  = &-\left(\sum_i \int_{N^u_i} D^{x_k} u \phi \dx
            + \sum_j \int_{N^v_j}  D^{x_k} v \phi \dx\right)\\
  = &-\left(\sum_i \int_{N^u_i} D^{x_k} \max(u, v) \phi \dx
            + \sum_j \int_{N^v_j}  D^{x_k} \max(u, v) \phi \dx\right)\\
  = &-\int_{\Omega} D^{x_k} \max(u, v) \phi \dx
\end{align*}
thus proving our claim that $\max(u, v)$ has weak derivatives up to order $k$.

Next we show that for $W^{2,p}(\Omega)$, the previous does not necessarily hold.
Consider $\Omega = (-1, 1)$ with
\begin{align*}
  u = &x\\
  v = &-x
\end{align*}
Clearly, $\max(u, v) = |x|$, which isn't in $W^{2, p}(\Omega)$.

\item Consider the following function
$$
u(x) = \frac{1}{|x|^\gamma}
$$
in $\Omega = B_1(0)$.

Show that if $\gamma + 1 < n$, the weak derivatives are given by
$$
\partial_j u = -\gamma \frac{x_j}{|x|^{\gamma + 2}}
$$
i.e., you need to show rigorously that
$$
\int u \partial_j \phi = \int \phi \gamma \frac{x_j}{|x|^{\gamma + 2}}
$$

Find the condition on $\gamma$ s.t. $u \in W^{1, p}$ or $u \in W^{2, p}$

\item Let $\eta(t) = 1$ for $t \leq 0$ and $\eta(t) = 0$ for $t > 1$.
Let $f \in W^{k, p}(R^n)$ and $f_k = f_\eta(|x| - k)$.
Show that $||f_k - f||_{W^{k, p}} \rightarrow 0$ as $k \rightarrow \infty$.
As a consequence show that $W^{k, p}(R^n) = W_0^{k, p}(R^n)$.

\item Let $u \in C^\infty(\closure{R}_+^n)$. Extend $u$ to $E u$ on $R^n$ such that
\begin{align*}
E u = &u, x \in R_+^n\\
E u \in &C^{3, 1}(R^n) \cap W^{4, p}(R^n)\\
||E u||_{W^{4, p}} \leq &C ||u||_{W^{4, p}}
\end{align*}
Here $R_+^n = \set{(x', x_n); x_n > 0}$ and $C^{3, 1} = \set{u \in C^3, D^\alpha u \text{ is Lipschitz, } |\alpha| = 3}$

First we assume the form
\begin{align*}
E u(x', x_n) =
\begin{cases}
u(x', x_n) & x_n \geq 0\\
c_0 u(x', -x_n) + c_1 u(x', -x_n / 2) + c_2 u(x', -x_n / 4) + c_3 u(x', -x_n / 8) + c_4 u(x', -x_n / 16) & x_n \leq 0
\end{cases}
\end{align*}
while requiring $c_0 + c_1 + c_2 + c_3 + c_4 = 1$.
This clearly satisfies $E u(x', x_n) = u(x', x_n)$ when $x_n \geq 0$, and is continuous since
$\limitto{x_n}{0} E u(x', x_n) = u(x', 0)$.
We also have $E u \in C^\infty(R_-^n)$, and just need to show that $\limitto{x_n}{0} D^{x_n^k} E u$ exists for $k \leq 3$.

Then when $x_n > 0$, we have $D^{x_n^k} E u(x', x_n) = D^{x_n^k} u(x', x_n)$
For $x_n < 0$, we have
\begin{align*}
\limitto{x_n}{0, x_n < 0} D^{x_n^k} E u(x', x_n) =
  \limitto{x_n}{0} &(-1)^k c_0 D^{x_n^k} u(x', -x_n) + \left(\frac{-1}{2}\right)^k c_1 D^{x_n^k} u(x', -x_n / 2)\\
  &+ \left( \frac{-1}{4} \right)^k c_2 D^{x_n} u(x', -x_n / 4) + \left( \frac{-1}{8} \right)^k c_3 D^{x_n} u(x', -x_n / 8)\\
  &+ \left( \frac{-1}{16} \right)^k c_4 D^{x_n} u(x', -x_n / 16)\\
  = &\left[ (-1)^k c_0 + \left(\frac{-1}{2}\right)^k c_1 + \left( \frac{-1}{4} \right)^k c_2
     + \left( \frac{-1}{8} \right)^k c_3 + \left( \frac{-1}{16} \right)^k c_4 \right] D^{x_n} u(x', 0)
\end{align*}
Then for $k \in \set{0, 1, 2, 3, 4}$ we need
\begin{align*}
  1 = (-1)^k c_0 + \left(\frac{-1}{2}\right)^k c_1 + \left( \frac{-1}{4} \right)^k c_2
      + \left( \frac{-1}{8} \right)^k c_3 + \left( \frac{-1}{16} \right)^k c_4
\end{align*}
This is a nonsingular linear system; it has a unique solution of $c_0 = \frac{51}{7}$, $c_1 = -\frac{1020}{7}$,
$c_2 = -816$, $c_3 = -\frac{10880}{7}$, $c_4 = \frac{6144}{7}$.
This extension has 4 derivatives by construction, and thus, has at least 3 continuous derivatives, so it's in $C^{3, 1}$.

Next, we show that $||E u||_{W^{4, p}} \leq C ||u||_{W^{4, p}}$.
\begin{align*}
  \sum_{|\alpha| \leq 4} \int_{R^n} |D^\alpha E u|^p \dx
    = &\sum_{|\alpha| \leq 4} \int_{R^{n - 1} \times (0, \infty)} |D^\alpha u|^p \dx
      + \int_{R^{n - 1} \times (-\infty, 0)} |D^\alpha E u|^p \dx\\
    = &||u||_{W^{4, p}} + \sum_{|\alpha| \leq 4} \int_{R^{n - 1} \times (-\infty, 0)} |D^\alpha E u|^p \dx\\
    = &||u||_{W^{4, p}} + \sum_{|\alpha| \leq 4} \int_{R^{n - 1} \times (-\infty, 0)}
                          \left| D^\alpha \sum_{i = 0}^4 c_i u(x_1, x_2, \ldots, x_{n - 1}, -x_n / 2^i) \right|^p \dx\\
 \leq &||u||_{W^{4, p}} + \sum_{|\alpha| \leq 4} \int_{R^{n - 1} \times (0, \infty)}
                          C \left| \sum_{i = 0}^4 D^\alpha c_i u(x_1, x_2, \ldots, x_{n - 1}, x_n) \right|^p \dx
 \leq C ||u||_{W^{4, p}}
\end{align*}

\item Let $R^+ = \set{x \in R; x > 0}$ and assume that $u \in W^{2, p}(R^+)$.
Define the symmetric extension of $u$ by setting $Eu(x) = u(|x|)$.
Prove that $Eu \in W^{1, p}(R)$ but $Eu \notin W^{2, p}(R)$ in general.

\item \begin{enumerate}
\item
  If $n = 1$ and $u \in W^{1, 1}(\Omega)$ then show that $u \in L^\infty$ and $u$ is continuous.

  First, WLOG, assume $\Omega = \reals$, since we can extend $u$ to $W^{1, 1}(\reals)$ if it's not.
  Then by problem 7, we have $u \rightarrow 0$ as $x \rightarrow \pm \infty$.
  $u \in W^{1, 1}(\Omega)$, so $D u$ exists, $\int_\Omega |u| \dx$, $\int_\Omega |D u| \dx$ are finite.
  If we assume that $\Omega$ is $\reals$, we have $W^{1, p}(\Omega) = W^{1, p}_0(\Omega)$,
  and we can assume it $u$ compact support.
  Then, since we have $|u(x)| \leq \int_{-\infty}^x |u'| \dx \leq ||D u||_{L^1(\reals)}$, $u \in L^\infty(\reals)$.

  Next, we need to show that $u$ is continuous.
  Assume not, let $x_d$ be a point of discontinuity, $x_{min} = \inf(\Omega)$, $x_{max} = \sup(\Omega)$,
  $u^+ = \limitto{x}{x_d, x > x_d} u(x)$, $u^- = \limitto{x}{x_d, x < x_d} u(x)$.
  Then
  \begin{align*}
    \int_{\Omega} u(x) \phi'(x) \dx = &\int_{x_{min}}^{x_0} u(x) \phi'(x) \dx + \int_{x_0}^{x_{max}} u(x) \phi'(x) \dx\\
                                   = &u^- \phi(x_d) - u(x_{min}) \phi(x_{min}) - \int_{x_{min}}^{x_0} D u(x) \phi(x) \dx\\
                                      &+ u(x_{max}) \phi(x_{max}) - u^+ \phi(x_{d}) - \int_{x_{min}}^{x_0} D u(x) \phi(x) \dx\\
                                   = &(u^- - u^+) \phi(x_d) + u(x_{max}) \phi(x_{max}) - u(x_{min}) \phi(x_{min})\\
                                      &- \int_{x_{min}}^{x_0} D u(x) \phi(x) \dx - \int_{x_{min}}^{x_0} D u(x) \phi(x) \dx
  \end{align*}
  As we did in problem 9, let $\phi_m$ be a monotonic sequence with $\phi_m(x_d) \neq 0$
  and $\phi_m(x) \rightarrow 0$ for $x \neq x_d$.
  Then $\int_\Omega D u(x) \phi_m(x) \dx \rightarrow 0 \neq (u^- - u^+) \phi(x_d)$,
  contradicting the claim that $u$ has a discontinuity.

\item
  If $n > 1$, find an example of $u \in W^{1, n}(B_1)$ and $u \notin L^\infty$

  The previous functions don't work, becuase $\gamma < 0$ puts the function in $L^\infty$.
  Example 4 from Evans doesn't work, since that requires $\gamma \in \emptyset$.

\end{enumerate}

\end{enumerate}
\end{document}
