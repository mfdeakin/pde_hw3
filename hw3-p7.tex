Let $\eta \in C^p(\Omega)$ with $\eta(t) = 1$ for $t \leq 0$ and $\eta(t) = 0$ for $t > 1$.
Let $f \in W^{k, p}(R^n)$ and $f_m = f(x) \eta(|x| - m)$.
Show that $||f_m - f||_{W^{k, p}} \rightarrow 0$ as $m \rightarrow \infty$.
As a consequence show that $W^{k, p}(R^n) = W_0^{k, p}(R^n)$.

We want to show that
\begin{align*}
0 = &\lim_{k \rightarrow \infty} \sum_{|\alpha| \leq p} \int_{x \in R^n} \left| D^\alpha (f(x) - f(x) \eta(|x| - k)) \right|^p \dx\\
  = &\sum_{|\alpha| \leq p} \lim_{k \rightarrow \infty} \int_{x \in B_k^C} \left| D^\alpha f(x) \right|^p \dx
\end{align*}

For $k = 0$, we can just apply the monotone convergence theorem,
since $|f - f_m|^p$ converges monotonically to $0$, showing that
$\limitto{m}{\infty} \int_{R^n} |f - f_m|^p \dx = \int_{R^n} \limitto{m}{\infty} |f - f_m|^p \dx = 0$.
For $p \geq 1$, we need to consider $\int_{R^n} |D^\alpha (f - f_m)|^p \dx$.
Breaking this into three integrals, we have
\begin{align*}
||f - f_m||_{W^{k, p}} = \sum_{|\alpha| = k} &\int_{|x| < m} [D^\alpha 0]^p \dx\\
                                        &+ \int_{m < |x| < m + 1} [D^\alpha [f(x) (1 - \eta(|x| - m))]]^p \dx\\
                                        &+ \int_{m + 1 < |x|} [D^\alpha f(x)]^p \dx\\
                     = \sum_{|\alpha| = k} &\int_{m < |x| < m + 1} [D^\alpha [f(x) (1 - \eta(|x| - m))]]^p \dx\\
                                        &+ \int_{R^n} [D^\alpha f(x)]^p \dx - \int_{|x| < m + 1} [D^\alpha f(x)]^p \dx
\end{align*}

We note that $\limitto{m}{\infty} \int_{R^n} [D^\alpha f(x)]^p \dx - \int_{|x| < m + 1} [D^\alpha f(x)]^p \dx = 0$,
since $f(x) \in W^{k, p}(R^n)$.
Then we're left with
\begin{align*}
\limitto{m}{\infty} ||f - f_m||_{W^{k, p}}
  = \sum_{|\alpha| = k} \limitto{m}{\infty} \int_{m < |x| < m + 1} [D^\alpha [f(x) (1 - \eta(|x| - m))]]^p \dx
\end{align*}

We just need to show that this goes to zero.
Let $S_\alpha = \set{(\beta, \gamma) | \alpha_i = \beta_i + \gamma_i, \beta_i, \gamma_i \geq 0}$
Then we can rewrite the weak derivative above with the product rule:
$$
D^\alpha [f(x) (1 - \eta(|x| - m))]
  = \sum_{(\beta, \gamma) \in S_\alpha} [D^\beta f(x)] D^{\gamma}(1 - \eta(|x| - m)) \dx
  \leq C \sum_{(\beta, \gamma) \in S_\alpha} D^\beta f(x)
$$

Then
\begin{align*}
\limitto{m}{\infty} ||f - f_m||_{W^{k, p}}
  \leq &\limitto{m}{\infty} \sum_{|\alpha| = k} \int_{m < |x| < m + 1} [D^\alpha [f(x) (1 - \eta(|x| - m))]]^p \dx\\
  \leq &\limitto{m}{\infty} \sum_{|\alpha| = k} \int_{m < |x|} C D^\alpha f(x) \dx = 0
\end{align*}
completing our proof.

Next, we note that $f_m \in W_{0}^{k, p}(R^n)$, so
$\forall \epsilon > 0, \exists g_i \in C_c^\infty(R^n)$ s.t. $|f_m - g_i| < \frac{\epsilon}{2}$,
and also we can choose $M \in N$ s.t. $\forall m \geq M$, $|f_m - f| < \frac{\epsilon}{2}$.
Through the triangle inequality we then have $g_i \rightarrow f$.
Thus, $f \in W_0^{k, p}(R^n)$.
Since this is true for any $f \in W^{k, p}(R^n)$, and since $W_0^{k, p}(R^n) \subset W^{k, p}(R^n)$,
we have $W^{k, p}(R^n) = W_0^{k, p}(R^n)$.

%% Then we note that since $f \in W^{k, p}(R^n)$, 
%% $S = \sum_{|\alpha| \leq p} \int_{R^n} \left| D^\alpha f(x) \right|^p \dx$ is finite.
%% % See proposition 1.12 of Stein and Shakarchi's Real Analysis text; pg 65
%% % Monotone convergence theorem FTW!!!
%% By the Fatou's theorem, since $\lim_{k \rightarrow \infty} |D^\alpha f_k| = |D^\alpha f|$,
%% we have $\int_{R^n} |D^\alpha f| \dx \leq \liminf_{k \rightarrow \infty} \int_{R^n} |D^\alpha f_k| \dx$.

%% $$
%% \limitto{N}{\infty} \int_{R^n} |D^\alpha f_N|^p \dx = \int_{R^n} \limitto{N}{\infty} |D^\alpha f_N|^p \dx = \int_{R^n} |D^\alpha f|^p \dx
%% $$
%% Then
%% $$
%% \limitto{N}{\infty} \int_{R^n} |f - f_N|^p \dx = \int_{R^n} \limitto{N}{\infty} |f_N|^p \dx = \int_{R^n} |f|^p \dx
%% $$


%% So for any $\epsilon \geq 0$, consider $\Omega_\epsilon = \set{x : |f(x)|^p > \epsilon}$.
%% Then
%% \begin{align*}
%%   S = &\int_{\Omega_\epsilon} |f(x)|^p \dx + \int_{\Omega_\epsilon^C} |f(x)|^p \dx
%%     \geq |\Omega_\epsilon| \epsilon + \int_{\Omega_\epsilon^C} |f(x)|^p \dx\\
%%   0 \leq &\int_{\Omega_{\epsilon}^C} |f(x)|^p \dx \leq S - |\Omega_\epsilon| \epsilon
%% \end{align*}
%% Thus, $|\Omega_\epsilon|$ is finite.
%% Next, write $\Omega_\epsilon = \bigcup_i \Omega_\epsilon^i$, where $\set{\Omega_\epsilon^i}$
%% consists of disconnected sets.

%% We have $|\Omega_\epsilon| = \sum_i |\Omega_\epsilon^i|$, showing this is an absolutely convergent sequence.
%% Thus, for all $\delta > 0$, we can choose $N > 0$ s.t. $|\Omega_\epsilon| - \sum_i^N |\Omega_\epsilon^i| < \delta$,
%% and
%% \begin{align*}
%%   \int_{\Omega_\epsilon} |f(x)|^p \dx \leq &\epsilon \delta + \sum_i^N \int_{\Omega_\epsilon^i} |f(x)|^p \dx\\
%%   S = &\int_{\Omega_\epsilon} |f(x)|^p \dx + \int_{\Omega_\epsilon^C} |f(x)|^p \dx
%%     \leq \epsilon \delta + \sum_i^N \int_{\Omega_\epsilon^i} |f(x)|^p \dx + \int_{\Omega_\epsilon^C} |f(x)|^p \dx\\
%%     \leq &\epsilon \delta + \sum_i^N \int_{\Omega_\epsilon^i} |f(x)|^p \dx + S - |\Omega_\epsilon| \epsilon\\
%%   |\Omega_\epsilon| \epsilon \leq &\sum_i^N \int_{\Omega_\epsilon^i} |f(x)|^p \dx
%% \end{align*}
